\newcommand\competencylevel{
	\begin{tabularx}{\linewidth}{|X|C{\tableGradeWidthLevel}|C{\tableGradeWidth}|C{\tableGradeWidth}|C{\tableGradeWidth}|C{\tableGradeWidth}|}
		\hline
		\textbf{Lernziele\,/\,Kompetenzen des Fachbereiches} & \rot{\parbox{3cm}{\thead{Niveaustufe\\überwiegend}}} & \rot{\thead{sehr gut erfüllt}} & \rot{\thead{gut erfüllt}} & \rot{\thead{teilweise}} & \rot{\thead{nicht erfüllt}} \\
		\hline
	\end{tabularx}
}

\newcommand{\zeugnisDay}{11.02.2022}

\newcommand{\gradeOne}{& $\boxtimes$ & $\Box$ & $\Box$ & $\Box$\\\hline}
\newcommand{\gradeTwo}{& $\Box$ & $\boxtimes$ & $\Box$ & $\Box$\\\hline}
\newcommand{\gradeThree}{& $\Box$ & $\Box$ & $\boxtimes$ & $\Box$\\\hline}
\newcommand{\gradeFour}{& $\Box$ & $\Box$ & $\Box$ & $\boxtimes$\\\hline}
\newcommand{\gradeNotGiven}{& \multicolumn{4}{c|}{nicht erteilt}\\\hline}
\newcommand{\gradeComesWithSecondHalfYear}{& \multicolumn{4}{c|}{wird im 2. Halbjahr belegt}\\\hline}
\newcommand{\gradeDefault}{&  &  &  & \\\hline}


\newcommand\levelRow[1]{\phantom{PH} & #1 & \multicolumn{4}{c|}{}\\\hline}
\newcommand{\levelOne}{\multicolumn{1}{m{\tableGradeWidthLevel}|}{\centering\color{greenEnglish} grün}}
\newcommand{\levelTwo}{\multicolumn{1}{m{\tableGradeWidthLevel}|}{\centering\color{blue} blau}}
\newcommand{\levelThree}{\multicolumn{1}{m{\tableGradeWidthLevel}|}{\centering\color{red} rot}}
\newcommand{\levelSeven}{\multicolumn{1}{m{\tableGradeWidthLevel}|}{\centering\tiny bis Klasse 7 ohne Niveau}}
\newcommand{\levelEight}{\multicolumn{1}{m{\tableGradeWidthLevel}|}{\centering\tiny bis Klasse 8 ohne Niveau}}
\newcommand{\levelNine}{\multicolumn{1}{m{\tableGradeWidthLevel}|}{\centering\tiny bis Klasse 9 ohne Niveau}}
\newcommand{\noLevel}{\multicolumn{1}{m{\tableGradeWidthLevel}|}{}}

%\setlength{\extrarowheight}{0pt}
\newcommand\competencyMajorSubject[2]{\makecell[{{>{\parindent0em}p{\linewidth}}}]{#1} & #2}
\newcommand\competencyMinorSubject[3]{\makecell[{{{>{\parindent0em}p\linewidth}}}]{#1} & #3 #2}

\newcommand\competencyTableMajorSubject[3]{\small
	\renewcommand{\arraystretch}{1.5}
	\begin{tabularx}{\linewidth}{|K{X}|C{\tableGradeWidthLevel}|C{\tableGradeWidth}|C{\tableGradeWidth}|C{\tableGradeWidth}|C{\tableGradeWidth}|}
		\hline
		\multicolumn{6}{|>{\columncolor{egelightblue}}l|}{\textit{\textbf{#1}}} \\
		\hline
		#2
		\levelRow{#3}
	\end{tabularx}\vspace{2pt}
}
\newcommand\competencytable[2]{\small
	\begin{tabularx}{\linewidth}{|K{X}|C{\tableGradeWidthLevel}|C{\tableGradeWidth}|C{\tableGradeWidth}|C{\tableGradeWidth}|C{\tableGradeWidth}|}
		\hline
		\multicolumn{6}{|>{\columncolor{egelightblue}}l|}{\textit{\textbf{#1}}} \\
		\hline
		#2
	\end{tabularx}\vspace{2pt}
}


\newcommand{\newpagedefs}{
	\newpage
	\newgeometry{headheight=106pt,top=7cm,
		bottom=2cm,
		inner=2cm,
		outer=2cm}
	\setlength{\headsep}{-0.4cm}
	\pagestyle{fancy}
}


\pagestyle{fancy}
\chead{\competencylevel}

\fancypagestyle{mypagestyle}{%
	\chead{\competencylevel}
	\cfoot{\small\raggedright\textit{Legende}:	Niveau Grün = Hauptschule ( AE I ), Niveau Blau = Realschule ( AE II ), Niveau Rot = Gymnasium ( AE III ),	n.b. = nicht bewertet}
}
\renewcommand{\headrulewidth}{0pt}

\newcommand{\tableGradeWidth}{.050\linewidth}
\newcommand{\tableLineBreak}{\vspace{.2em}\newline}