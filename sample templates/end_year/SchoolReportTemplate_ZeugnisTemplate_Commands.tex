\newcommand\competencylevel{
	\begin{tabularx}{\linewidth}{|X|C{\tableGradeWidth}|C{\tableGradeWidth}|C{\tableGradeWidth}|C{\tableGradeWidth}|}
		\hline
		\textbf{Kompetenzfelder und Lernziele der Fachbereiche} & \rot{\parbox{2.7cm}{\thead{sehr gut erfüllt}}} & \rot{\thead{gut erfüllt}} & \rot{\thead{teilweise}} & \rot{\thead{nicht erfüllt}} \\
		\hline
	\end{tabularx}
}

\newcommand{\gradeOne}{& $\boxtimes$ & $\Box$ & $\Box$ & $\Box$\\\hline}
\newcommand{\gradeTwo}{& $\Box$ & $\boxtimes$ & $\Box$ & $\Box$\\\hline}
\newcommand{\gradeThree}{& $\Box$ & $\Box$ & $\boxtimes$ & $\Box$\\\hline}
\newcommand{\gradeFour}{& $\Box$ & $\Box$ & $\Box$ & $\boxtimes$\\\hline}
\newcommand{\gradeNotGiven}{& \multicolumn{4}{c|}{nicht erteilt}\\\hline}
\newcommand{\gradeComesWithSecondHalfYear}{& \multicolumn{4}{c|}{wird im 2. Halbjahr belegt}\\\hline}
\newcommand{\gradeDefault}{&  &  &  & \\\hline}


\newcommand\levelRow[1]{\multicolumn{5}{|c|}{#1}\\\hline}
\newcommand{\levelOne}{\levelRow{\centering Du hast vorwiegend auf Niveau {\color{greenEnglish} Anforderungsebene 1} gearbeitet.}}
\newcommand{\levelTwo}{\levelRow{\centering Du hast vorwiegend auf Niveau {\color{blue} Anforderungsebene 2} gearbeitet.}}
\newcommand{\levelThree}{\levelRow{\centering Du hast vorwiegend auf Niveau {\color{red} Anforderungsebene 3} gearbeitet.}}
\newcommand{\levelSeven}{\levelRow{\centering bis Klasse 7 ohne Niveau}}
\newcommand{\levelEight}{\levelRow{\centering bis Klasse 8 ohne Niveau}}
\newcommand{\levelNine}{\levelRow{\centering bis Klasse 9 ohne Niveau}}
\newcommand{\noLevel}{}


% #1 ... competence
% #2 ... grade
\newcommand\competencyMajorSubject[2]{\makecell[{{>{\parindent0em}p{\linewidth}}}]{#1} #2}

% #1 ... subject
% #2 ... competencies
% #3 ... level
\newcommand\competencyTableMajorSubject[3]{
	\renewcommand{\arraystretch}{1.5}
	\begin{tabularx}{\linewidth}{|K{X}|C{\tableGradeWidth}|C{\tableGradeWidth}|C{\tableGradeWidth}|C{\tableGradeWidth}|}
		\hline
		\multicolumn{5}{|>{\columncolor{egelightblue}}l|}{\textit{\textbf{#1}}} \\
		\hline
		#2
		#3
	\end{tabularx}\vspace{2pt}
}


\newcommand{\newpagedefs}{
	\newpage
	\newgeometry{headheight=106pt,top=7cm,
		bottom=2cm, 
		inner=2cm,
		outer=2cm}
	\setlength{\headsep}{-0.4cm}
	\pagestyle{fancy}
}

\newcommand{\newpagedefsLastPage}{
	\newpage
	\newgeometry{top=2cm,
		bottom=2cm, 
		inner=2cm,
		outer=2cm}
	\setlength{\headsep}{-0.4cm}
	\pagestyle{fancy}
}

\newcommand\formatText[1]{
	\noexpandarg
	\StrBefore{#1}{\\}[\studentIntro]
	\StrBehind{#1}{\\}[\certText]
	{\Large\setstretch{1.5}\textbf{\LARGE\studentIntro\vspace{.5em}\\}\certText\par}
}

\pagestyle{fancy}
\chead{\competencylevel}

\fancypagestyle{mypagestyle}{%
	\chead{\competencylevel}
	\cfoot{\small\raggedright\textit{Legende}:	Niveau Grün = Hauptschule ( AE I ), Niveau Blau = Realschule ( AE II ), Niveau Rot = Gymnasium ( AE III ),	n.b. = nicht bewertet}
}
\renewcommand{\headrulewidth}{0pt}

\newcommand{\tableGradeWidth}{.050\linewidth}
\newcommand{\tableLineBreak}{\vspace{.2em}\newline}