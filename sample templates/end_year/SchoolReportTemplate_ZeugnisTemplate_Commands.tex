\makeatletter
\def\nobreakhline{%
	\noalign{\ifnum0=`}\fi
	\penalty\@M
	\futurelet\@let@token\LT@@nobreakhline}
\def\LT@@nobreakhline{%
	\ifx\@let@token\hline
	\global\let\@gtempa\@gobble
	\gdef\LT@sep{\penalty\@M\vskip\doublerulesep}% <-- change here
	\else
	\global\let\@gtempa\@empty
	\gdef\LT@sep{\penalty\@M\vskip-\arrayrulewidth}% <-- change here
	\fi
	\ifnum0=`{\fi}%
	\multispan\LT@cols
	\unskip\leaders\hrule\@height\arrayrulewidth\hfill\cr
	\noalign{\LT@sep}%
	\multispan\LT@cols
	\unskip\leaders\hrule\@height\arrayrulewidth\hfill\cr
	\noalign{\penalty\@M}%
	\@gtempa}
\makeatother


\newcommand\competencyLevel{
	\begin{tabularx}{\linewidth}{|X|C{\tableGradeWidth}|C{\tableGradeWidth}|C{\tableGradeWidth}|C{\tableGradeWidth}|}
		\hline
		\textbf{Kompetenzfelder und Lernziele der Fachbereiche} & \rot{\parbox{2.7cm}{\thead{sehr gut erfüllt}}} & \rot{\thead{gut erfüllt}} & \rot{\thead{teilweise}} & \rot{\thead{nicht erfüllt}} \\
		\hline
	\end{tabularx}
}

\newcommand{\gradeOne}{& $\boxtimes$ & $\Box$ & $\Box$ & $\Box$\\}
\newcommand{\gradeTwo}{& $\Box$ & $\boxtimes$ & $\Box$ & $\Box$\\}
\newcommand{\gradeThree}{& $\Box$ & $\Box$ & $\boxtimes$ & $\Box$\\}
\newcommand{\gradeFour}{& $\Box$ & $\Box$ & $\Box$ & $\boxtimes$\\}
\newcommand{\gradeNotGiven}{& \multicolumn{4}{c|}{nicht erteilt}\\}
\newcommand{\gradeComesWithSecondHalfYear}{& \multicolumn{4}{c|}{wird im 2. Halbjahr belegt}\\}
\newcommand{\gradeDefault}{&  &  &  & \\}


\newcommand\levelRow[1]{\multicolumn{5}{|c|}{#1}\\\hline}
\newcommand{\levelOne}{\levelRow{\centering Du hast vorwiegend auf {\color{greenEnglish} Anforderungsebene 1} gearbeitet.}}
\newcommand{\levelTwo}{\levelRow{\centering Du hast vorwiegend auf {\color{blue} Anforderungsebene 2} gearbeitet.}}
\newcommand{\levelThree}{\levelRow{\centering Du hast vorwiegend auf {\color{red} Anforderungsebene 3} gearbeitet.}}
\newcommand{\levelSeven}{\levelRow{\centering bis Klasse 7 ohne Anforderungsebene}}
\newcommand{\levelEight}{\levelRow{\centering bis Klasse 8 ohne Anforderungsebene}}
\newcommand{\levelNine}{\levelRow{\centering bis Klasse 9 ohne Anforderungsebene}}
\newcommand{\noLevel}{}


% #1 ... competence
% #2 ... grade
\newcommand\competencyMajorSubject[2]{\makecell[{{>{\parindent0em}p{\linewidth}}}]{#1} #2}

% #1 ... subject
% #2 ... competencies
% #3 ... level
\newcommand\competencyTableMajorSubject[3]{
	\renewcommand{\arraystretch}{1.5}
	\begin{xltabular}{\linewidth}{|K{X}|C{\tableGradeWidth}|C{\tableGradeWidth}|C{\tableGradeWidth}|C{\tableGradeWidth}|}
		\nobreakhline
		\multicolumn{5}{|>{\columncolor{egelightblue}}l|}{\textit{\textbf{#1}}}\\
		\nobreakhline
		#2
		#3
	\end{xltabular}\vspace{2pt}
}

% argument is the page threshold until a newline should be added
\newcommand\newOptionalNewPage[1]{
	\ifnum #1>\value{page}
		\newpage
		\thispagestyle{plain}
		\phantom{~}
	\fi
}

\newcommand{\newpagedefs}{
	\newpage
	\newgeometry{headheight=106pt,top=7cm,
		bottom=2cm, 
		inner=2cm,
		outer=2cm}
	\setlength{\headsep}{-0.4cm}
	\pagestyle{fancy}
}

\newcommand{\newpagedefsLastPage}{
	\newpage
	\newgeometry{top=2cm,
		bottom=2cm, 
		inner=2cm,
		outer=2cm}
	\pagestyle{fancy}
}

\newcommand\formatText[1]{
	\noexpandarg
	\StrBefore{#1}{\\}[\studentIntro]
	\StrBehind{#1}{\\}[\certText]
	{\Large\setstretch{1.15}\textbf{\LARGE\studentIntro\vspace{.5em}\\}\certText\par}
}

\pagestyle{fancy}
\chead{\competencyLevel}

\renewcommand{\headrulewidth}{0pt}