\documentclass[11pt,a4paper]{article}
\usepackage[top=2cm,
bottom=2cm, 
inner=2cm,
outer=2cm]{geometry}
\usepackage{xcolor}
\definecolor{egeblue}{RGB}{00, 99, 142}
\definecolor{egelightblue}{RGB}{66, 144, 179}


\usepackage[german]{babel}
\usepackage[T1]{fontenc}
\usepackage[utf8]{inputenc}

\usepackage{ragged2e}
\usepackage{titlesec}

\usepackage{amsmath}
\usepackage{amssymb}
\usepackage{helvet}

\usepackage{fancyhdr}
\usepackage{graphicx}

\usepackage{array,multirow,tabularx,booktabs}
\usepackage[column=K]{cellspace}
\usepackage{makecell,colortbl}
\newcolumntype{Y}{>{\centering\arraybackslash}X}
\newcolumntype{L}[1]{>{\raggedright\arraybackslash}p{#1}}
\newcolumntype{C}[1]{>{\centering\arraybackslash}p{#1}}
\newcolumntype{R}[1]{>{\raggedleft\arraybackslash}p{#1}}
\newcolumntype{J}[1]{>{\justifying\arraybackslash}p{#1}}
\newcolumntype{v}[1]{>{\raggedright\hspace{0em}}p{#1}}

\newcommand*\rot{\rotatebox{90}}


\renewcommand\theadalign{l}
\renewcommand\theadfont{\bfseries}

% package for linespacing \singlespacing \onehalfspacing \doublespacing and more
\usepackage{setspace}
% package to enable/disable hyphenation and set hyphenations
\usepackage{hyphenat}
% provides enumarate environment
\usepackage{enumitem}
% package for table and figure captions
\usepackage{caption}
%% setups for table and graphic captions
% package sto change title styles (sections etc)

% renews command to use sans font as default font
\renewcommand*{\familydefault}{\sfdefault}

\pagenumbering{gobble}
\addparagraphcolumntypes{X}

\setlength\parindent{0pt}
\setlength\cellspacetoplimit{8pt}
\setlength\cellspacebottomlimit{8pt}

\newcommand{\tableGradeWidth}{.050\linewidth}

\newcommand\competencylevel{
	\begin{tabularx}{\linewidth}{|X|C{.07\linewidth}|C{\tableGradeWidth}|C{\tableGradeWidth}|C{\tableGradeWidth}|C{\tableGradeWidth}|}
		\hline
		\textbf{Lernziele\,/\,Kompetenzen des Fachbereiches} & \rot{\parbox{3cm}{\thead{Niveaustufe\\überwiegend}}} & \rot{\thead{sehr gut erfüllt}} & \rot{\thead{gut erfüllt}} & \rot{\thead{teilweise}} & \rot{\thead{nicht erfüllt}} \\
		\hline
	\end{tabularx}
}

\newcommand{\zeugnisDay}{11.02.2022}

\newcommand{\gradeOne}{& $\boxtimes$ & $\Box$ & $\Box$ & $\Box$\\\hline}
\newcommand{\gradeTwo}{& $\Box$ & $\boxtimes$ & $\Box$ & $\Box$\\\hline}
\newcommand{\gradeThree}{& $\Box$ & $\Box$ & $\boxtimes$ & $\Box$\\\hline}
\newcommand{\gradeFour}{& $\Box$ & $\Box$ & $\Box$ & $\boxtimes$\\\hline}
\newcommand{\gradeNon}{& \multicolumn{4}{c|}{nicht erteilt}\\\hline}
\newcommand{\gradeHj}{& \multicolumn{4}{c|}{wird im 2. Halbjahr belegt}\\\hline}
\newcommand{\gradeDefault}{&  &  &  & \\\hline}


\newcommand\levelRow[1]{\phantom{PH} & \multicolumn{1}{m{.07\linewidth}|}{\centering #1} & \multicolumn{4}{c|}{}\\\hline}
\newcommand{\levelSeven}{\multicolumn{1}{m{.07\linewidth}|}{\centering\tiny bis Klasse 7 ohne Niveau}}
\newcommand{\levelEight}{\multicolumn{1}{m{.07\linewidth}|}{\centering\tiny bis Klasse 8 ohne Niveau}}
\newcommand{\levelNine}{\multicolumn{1}{m{.07\linewidth}|}{\centering\tiny bis Klasse 9 ohne Niveau}}

\newcommand\competencyMS[2]{\makecell[l]{#1} & #2}
\newcommand\competencySS[3]{\makecell[l]{#1} & #3 #2}

\newcommand\competencytableMS[3]{
	\renewcommand{\arraystretch}{1.5}
	\begin{tabularx}{\linewidth}{|K{X}|C{.07\linewidth}|C{\tableGradeWidth}|C{\tableGradeWidth}|C{\tableGradeWidth}|C{\tableGradeWidth}|}
		\hline
		\multicolumn{6}{|>{\columncolor{egelightblue}}l|}{\textit{\textbf{#1}}} \\
		\hline
		#2
		\levelRow{#3}
	\end{tabularx}\vspace{2pt}
}
\newcommand\competencytable[2]{
	\begin{tabularx}{\linewidth}{|K{X}|C{.07\linewidth}|C{\tableGradeWidth}|C{\tableGradeWidth}|C{\tableGradeWidth}|C{\tableGradeWidth}|}
		\hline
		\multicolumn{6}{|>{\columncolor{egelightblue}}l|}{\textit{\textbf{#1}}} \\
		\hline
		#2
	\end{tabularx}\vspace{2pt}
}


\newcommand{\newpagedefs}{
	\newpage
	\newgeometry{headheight=106pt,top=7cm,
		bottom=2cm, 
		inner=2cm,
		outer=2cm}
	\setlength{\headsep}{-0.4cm}
	\pagestyle{fancy}
}


\pagestyle{fancy}
\chead{\competencylevel}

\fancypagestyle{mypagestyle}{%
	\chead{\competencylevel}
	\cfoot{\small\raggedright\textit{Legende}:	Niveau Grün = Hauptschule ( AE I ), Niveau Blau = Realschule ( AE II ), Niveau Rot = Gymnasium ( AE III ),	n.b. = nicht bewertet}
}
\renewcommand{\headrulewidth}{0pt}

\begin{document}
	\pagestyle{plain}
	\fcolorbox{egeblue}{white}{
	\begin{minipage}[t][.98\textheight][t]{.97\textwidth}
			~\vspace{2cm}\\
			\begin{tabularx}{\linewidth}{@{}lX}
				Versäumnisse: & #absenceDays  Tage (davon #absenceUDays Tage unentschuldigt) \\
				&  #absenceHours Stunden (davon #absenceUHours Stunden unentschuldigt) \\
			\end{tabularx}
			\vspace{3cm}\\
			Erfurt, den \zeugnisDay
			\vspace{2.5cm}\\
			\begin{tabularx}{\linewidth}{C{.4\linewidth}XC{.4\linewidth}}
				\rule{\linewidth}{1pt} &  & \rule{\linewidth}{1pt} \\
				\centering KlassenleiterIn &  & KlassenleiterIn \\
			\end{tabularx}
			\vspace{2.5cm}\\
			\begin{tabularx}{\linewidth}{C{.4\linewidth}YC{.4\linewidth}}
				\rule{\linewidth}{1pt} &  &  \\
				\centering SchuleiterIn & \tiny Siegel & \\
			\end{tabularx}
			\vspace{2.5cm}\\
			\begin{tabularx}{\linewidth}{C{.4\linewidth}XC{.4\linewidth}}
			\rule{\linewidth}{1pt} &  & \rule{\linewidth}{1pt} \\
			\centering Erziehungsberechtigte &  & SchülerIn \\
			\end{tabularx}
		\end{minipage}
	}%
	\newpage
	\fcolorbox{egeblue}{white}{
		\begin{minipage}[t][.98\textheight][t]{.97\textwidth}
		\begin{center}\vspace{1cm}
			\includegraphics[width=.9\linewidth]{Logotop_mit_ESM_zeugnis}
		\end{center}
		\vspace{3cm}
		\begin{center}\setstretch{1.5}
			\textbf{\Huge Zeugnis}\\
			\textbf{\LARGE ~\\
			Evangelischen Gemeinschaftsschule Erfurt\\\vspace{2cm}
			#partOfYear #schoolYear}
		\end{center}\vspace{3cm}
		\begin{center}\LARGE
			\begin{tabular}{ll}
			\textbf{Klasse:} &  #schoolClass\vspace{1cm}\\
			\textbf{Name:} &  #givenName #surName\\
			& geb.: #birthDay
		\end{tabular}
		\end{center}
	\end{minipage}
	}%
	\newpagedefs

	#tables

	\thispagestyle{mypagestyle}
\end{document}





% Choose the document class whose layout you want to visualize: uncomment
% the one you want, comment out the others.
% \documentclass[a4paper]{article} %Produces one page (based on A4 paper size)
% \documentclass[a4paper]{report} %Produces one page (based on A4 paper size)
% \documentclass[twoside,a4paper]{report} %Produces two pages (based on A4 paper size)
% \documentclass[a4paper]{book} %Produces two pages (based on A4 paper size)
% \documentclass[a4paper]{letter} %Produces one page (based on A4 paper size)
% \documentclass[twoside, a4paper]{letter} %Produces two pages (based on A4 paper size)
%\documentclass[twoside,landscape,a3paper]{article} %Produces two pages (based on A4 paper size)
%%\usepackage{layout}
%\begin{document}
%%	\layoutaa
%aa
%\end{document}











